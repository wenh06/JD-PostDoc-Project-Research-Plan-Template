\documentclass{ctexart}

%-----------------------------------------------------
% weird problem would happen if setting these fonts

% \setCJKmainfont{simfang.ttf}
% \setCJKsansfont{simhei.ttf}
%-----------------------------------------------------

% %% Language and font encodings
% \usepackage[UTF8]{ctex}
% \usepackage[english]{babel}
% \usepackage[utf8x]{inputenc}

\usepackage{lipsum}

\usepackage[T1]{fontenc}
\usepackage{footmisc}

%% Sets page size and margins
\usepackage[a4paper,top=2.4cm,bottom=2.4cm,left=2.7cm,right=2.7cm,marginparwidth=1.75cm]{geometry}

% %% Useful packages
\usepackage{amsmath,amsfonts,amssymb,mathrsfs}
\usepackage{graphicx}
\usepackage{pst-func}
\usepackage{float}
% \usepackage{MnSymbol}
% \usepackage[colorinlistoftodos]{todonotes}
\usepackage[colorlinks=true, allcolors = blue]{hyperref}
\usepackage{tikz}
\usetikzlibrary{shapes,arrows,decorations.pathmorphing,backgrounds,positioning,fit,petri,calc,external}
\usepackage{url}

\usepackage{subcaption}
\usepackage{cleveref}
\captionsetup[table]{skip = 3pt}
\captionsetup[subfigure]{subrefformat=simple}

\allowdisplaybreaks

% section title left align
\makeatletter
\g@addto@macro{\CTEX@section@format}{\raggedright}
\makeatother

% Chinese section number
\usepackage{zhnumber}
\renewcommand\thesection{\zhnum{section}}
% \renewcommand{\thesubsection}{\thesection.\zhnum{subsection}}

\usepackage{cases}
\usepackage{tabularx, ragged2e}
\usepackage{booktabs}
\usepackage{fix-cm}
\usepackage{longtable}
% \usepackage{supertabular}
% \usepackage{mdframed}

\usepackage{titlesec}
\titleformat{\section}[block]{\LARGE \bfseries \filcenter}{}{1em}{}
\usepackage{chngcntr}
\counterwithout{subsection}{section}
\titlelabel{\thetitle、}

% \renewcommand\tabularxcolumn[1]{>{\Centering}m{#1}}

\newcolumntype{M}[1]{>{\Centering\arraybackslash}m{#1}}

\newcolumntype{x}[1]{>{\raggedright}m{#1}}
\newcolumntype{z}[1]{>{\centering}m{#1}}
\newcommand{\tn}{\tabularnewline}

\renewcommand\tabularxcolumn[1]{>{\centering\arraybackslash}m{#1}} %% COMMENT
\usepackage{makecell}
\renewcommand\cellalign{lc}
\usepackage{cellspace}
\setlength\cellspacetoplimit{2pt}
\setlength\cellspacebottomlimit{2pt}
\addparagraphcolumntypes{x, }
\usepackage{caption}

\makeatletter
\newcommand\HUGE{\@setfontsize\Huge{35}{46}}
\makeatother

\usepackage{hyperref}
\hypersetup{
   colorlinks = true,
    linkcolor = black,
    filecolor = magenta,      
     urlcolor = black,
}

\DeclareMathOperator*{\argmin}{\arg\!\min}

\usepackage[ruled,linesnumbered]{algorithm2e}
\usepackage{listings}

\usepackage{footnote}
\usepackage{footmisc}

\newcommand{\astfootnote}[1]{
\let\oldthefootnote=\thefootnote
\setcounter{footnote}{0}
\renewcommand{\thefootnote}{\fnsymbol{footnote}}
\footnote{#1}
\let\thefootnote=\oldthefootnote
}

\tikzexternalize[prefix=tikz/,optimize command away=\includepdf]

%--------------------------------------------------------
% project name, name of the applicant, etc.

\title{京东博士后科研工作站\\ \vspace{0.4cm}
拟申报课题研究计划书}
\author{
\begin{tabular}{rc}
课题名称 & \phantom{cc}research plan title\phantom{cc} \\
\cmidrule(l){2-2}
项目申请人 & \phantom{cc}XXX\phantom{cc} \\
\cmidrule(l){2-2}
毕业学校 & \phantom{cc}XX大学\phantom{cc} \\
\cmidrule(l){2-2}
申报方向 &  \\
\cmidrule(l){2-2}
\end{tabular}
}

%-------------------------------------------------------

\zhnumsetup{time=Chinese}
\date{\zhtoday}

%-------------------------------------------------------
% set up the cover page

\makeatletter         
\def\@maketitle{
\vspace*{\fill}
\begin{center}
\includegraphics[width = 15cm]{images/logo.jpg} \\
{\HUGE \bfseries \sffamily \@title } \\ \vspace{3.9cm}
{\LARGE  \@author} \\ \vspace{3cm}
{\LARGE \@date}
\end{center}
\vspace*{\fill}}
\makeatother
%-------------------------------------------------------

\begin{document}

\clearpage\maketitle
\thispagestyle{empty}

% {\color{magenta}
% 需要修改:项目题目缩小,增加深度
% }

\newpage

\section{一、申请人基本情况}
\setcounter{page}{1}
\pagestyle{plain}


\large

% \setlength{\extrarowheight}{8pt}

\renewcommand{\arraystretch}{1.7}

\begin{table}[H]
\large
% \caption*{\LARGE \bfseries 一、申请人基本情况}
\begin{tabularx}{\textwidth}{|X|X|X|X|X|X|}
\hline
姓名 & XXX & 性别 & X & 出生年月 & XXXX.XX \\
\hline
学位 & X学博士 & 民族 & X & 毕业时间 & XXXX.XX \\
\hline
毕业院系 & \multicolumn{2}{c|}{XX大学XXXXXX系} & 联系电话 & \multicolumn{2}{c|}{XXXXXX} \\
\hline
研究领域 & \multicolumn{2}{c|}{XXXXXX} & 邮箱 & \multicolumn{2}{c|}{\href{mailto:aaaaa@gmail.com}{\nolinkurl{aaaaa@gmail.com}}} \\
\hline
\multicolumn{2}{|c|}{获得主要奖励} & \multicolumn{4}{p{\dimexpr0.667\textwidth-2\tabcolsep-2\arrayrulewidth}|}{} \\
\hline
\multicolumn{2}{|c|}{主要学术成果} & \multicolumn{4}{p{\dimexpr0.667\textwidth-2\tabcolsep-2\arrayrulewidth}|}{XXX and XXX. XXXXXXXXXXXXXXXXXXXXXX. Journal, XX(X): XXX–XXX, XXXX. DOI: XXXXXXXXXXXXXXXXXXXXXX.} \\
\hline
\multicolumn{2}{|c|}{教育背景} & \multicolumn{4}{p{\dimexpr0.667\textwidth-2\tabcolsep-2\arrayrulewidth}|}{{\bf 本科}:XXXX.XX~ 至~ XXXX.XX~ XX大学~ XXXXXX~ X学学士~ 《XXXXXXXXXXXX》\newline
{\bf 博士}:XXXX.XX~ 至~ XXXX.XX~ XX大学~ XXXXXX~ X学博士~ 《XXXXXXXXXXXXXXXXXX》\newline
{\bf 交换项目}:XXXX.XX~ 至~ XXXX.XX~ XX大学
} \\
\hline
\end{tabularx}
\end{table}


\section{二、项目具体情况}

\subsection{\Large 项目阐述}

\subsubsection*{\large 项目背景:}

\input{content/project_background}

\subsubsection*{\large 研究方法:}

\input{content/research_methods}

\subsubsection*{\large 拟采取的技术路线:}

\input{content/technical_route}

\subsubsection*{\large 可行性分析:}

\input{content/feasibility_analysis}

\subsection{项目创新点}

\lipsum[9-10]


\subsection{项目研究计划和预期进展}

\lipsum[11-12]


\subsection{项目预期成果}

\input{content/expected_outcome}

% add bib items in references.bib
% \bibliographystyle{plain}
% \bibliography{references}

\end{document}
